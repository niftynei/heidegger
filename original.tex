p 10...

But in what, then, does the playing in unison of the four ways of occasioning play? They let what is not yet present arrive into presencing. Accordingly they are unifiedly ruled over by a bringing that brings what presences into appearnace. Plato tells us what this bringing is in a sentence from the \textit{Symposium} (205b): \textit{h\-{e} gar toi ek tou m\-{e} onton eis to on ionti hot\-{o}ioun aitia pasa esti poi\-{e}sis}. ``Every occasion for whatever passes over and goes forward into presencing from that which is not presencing is \textit{poi\-{e}sis}, is bringing-forth[\textit{Her-vor-bringen}]."

It is of utmost importance that we think bringing-forth in its full scope and at the same time in the sense in which the Greeks thought it. Not only handcraft manufacture, not only artistic and poetical bringing into appearance and concrete imagery, is a bringing-forth, \textit{poi\-{e}sis}. \testit{Physis} also, the arising of something form out of itself, is a bringing-forth, \textit{poi\-{e}sis}. \textit{Physis} is indeed \textit{poi\-{e}sis} in the highest sense. For what presences by means of \textit{physis} has the bursting open belonging to bringing-forth, e.g., the bursting of a blossom into bloom, in itself (\textit{en heaut\-{o}i}). In contrast, what is brought forth by the artisan or the artist, e.g., the silver chalice, has the bursting open belonging to bringing-forth not in itself, but in another (\textit{en all\-{o}i}), in the craftsman or artist.

The modes of occasioning, the four causes, are at play, then, within bringing-forth. Through bringing-forth, the growing things of nature as well as whatever is completed through the crafts and the arts come at any given time to their appearance.

But how does bringing-forth happen, be it in nature or in handwork and art? What is bringing-forth in which the fourfold way of occasioning plays? Occasioning has to do with the presencing [\textit{Anwesen}] of that which at any given time comes to appearance in bringing-forth. Bringing-forth brings hither out of concealment forth into unconcealment. Bringing-forth comes to pass only insofar as something concealed comes into unconcealment. This coming rests and moves freely within what we call revealing [\textit{das Entbergen}]. The Greeks have the word \textit{al\-{e}theia} for revealing. The Romans translate this with \textit{veritas}. We say ``truth" and usually understand it as the correctness of an idea.



But where have we strayed to? We are questioning concerning technology, and we have arrived now at \textit{al\-{e}theia}, at revealing. What has the essence of technology to do with revealing? The answer: everything. For every bringing-forth is grounded in revealing. Bringing-forth, indeed, gathers within itself the four modes of occasioning -- causality -- and rules them throughout. Within its domain belong end and means, belongs instrumentality. Instrumentality is considered to be the fundamental characteristic of technology. If we inquire, step by step, into what technology, represented as means, actually is, then we shall arrive at revealing. The possibility of all productive manufacturing lies in revealing.

Technology is therefore no mere means. Technology is a way of revealing. If we give heed to this, then another whole realm for the essence of technology will open itself up to us. It is the realm of revealing, i.e., of truth.

This prospect strikes us as strange. Indeed, it should do so, should do so as persistently as possible and with so much urgency that we will finally take seriously the simple question of what the name ``technology" means. The word stems from the Greek. \textit{Technikon} means that which belongs to \textit{techn\-{e}}. We must observe two things with respect to the meaning of this word. One is that \textit{techn\-{e}} is the name not only for the activities and skills of the craftsman, but also for the arts of the mind and the fine arts. \textit{Techn\-{e}} belongs to bringing-forth, to \{poi\-{e}sis}; it is something poietic.

The other point that we should observe with regard to \textit{techn\-{e}} is even more important. From earliest times until Plato the word \textit{techn\-{e}} is linked with the word \textit{epist\-{e}m\-{e}}. Both words are names for knowing in the widest sense. They mean to be entirely at home in something, to understand and be expert in it. Such knowing provides an opening up. As an opening up it is a revealing. Aristotle, in a discussion of special importance (\textit{Nichomachean Ethics}, Bk. VI, chaps. 3 and 4), distinguishes between \textit{epist\-{e}m\-{e}} and \textit{techn\-{e}} and indeed with respect to what and how they reveal. \textit{Techn\-{e}} is a mode of \textit{al\-{e}theuein}. It reveals whatever does not bring itself forth and does not yet lie here before us, whatever can look and turn out now one way and now another. Whoever builds a house or a ship or forges a sacrificial chalice reveals what is to be brought forth, according to the perspectives of the four modes of occasioning. This revealing gathers together in advance the aspect and the matter of ship or house, with a view to the finished thing envisioned as completed, and from this gathering determines the manner of its construction. Thus what is decisive in \textit{techn\-{e}} does not lie at all in making and manipulating nor in the using of means, but rather in the aforementioned revealing. It is as revealing, and not as manufacturing, that \textit{techn\-{e}} is a bringing-forth.

Thus the clue to what the word \textit{techn\-{e}} means and to how the Greeks defined it leads us into the same context that opened itself to us when we pursued the question of what instrumentality as such in truth might be.

Technology is a mode of revealing. Technology comes to presence in the realm where revealing and unconcealment take place, where \textit{al\-{e}thia}, truth, happens.

In opposition to this definition of the essential domain of technology, one can object that it indeed holds for Greek thought and that at best it might apply to the techniques of the handcraftsman, but that it simply does not fit modern machine-powered technology. And it is precisely the latter and it alone that is the distrubing thing, that moves us t oask the question concerning technology per se. It is said that modern technology is something incomparably different from all earlier technologies because it is based on modern physics as an exact science. Meanwhile we have come to understand more clearly that the reverse holds true as well: Modern physics, as experimental, is dependent upon technical apparatus and upon progress in the building of apparatus. The establishing of this mutual relationship between technology and physics is correct. But it remains a merely historiographical establishing of facts and says nothing about that in which this mutual relationship is grounded. The decisive question still remains: Of what essence is modern technology that it happens to think of putting exact science to use?

What is modern technology? It too is a revealing. Only when we allow our attention to rest on this fundamental characteristic does that which is new in modern technology show itself to us. 

And yet the revealing that holds sway throughout modern technology does not unfold into a bringing-forth in the sense of \textit{po\-{e}sis}. The revealing that rules in modern technology is a challenging, which puts to nature the unreasonable demand that it supply energy that can be extracted and stored as such. But does this not hold true for the old windmill as well? No. Its sails do indeed turn in the wind; they are left entirely to the wind's blowing. But the windmill does not unlock energy from the air currents in order to store it.

In contrast, a tract of land is challenged into the putting out of coal and ore. The earth now reveals itself as a coal mining district, the soil as a mineral desposit. The field that the peasant formerly cultivated and set in order appears differently than it did when to set in order still meant to take care of and to maintain. The work of the peasant does not challenge the soil of the field. In the sowing of the grain it places the seed in the keeping of the forces of growth and watches over its increase. But meanwhile even the cultivation of the field has come under the grip of another kind of setting-in-order, which \textit{sets} upon nature. It sets upon it in the sense of challenging it. Agriculture is now the mechanized food industry. Air is now set upon to yield nitrogen, the earth to yield ore, ore to yield uranium, for example; uranium is set upon to yield atomic energy, which can be released either for destruction or for peaceful use.

This setting-upon that challenges forth the energies of nature is an expediting, and in two ways. It expedites in that it unlocks and exposes. Yet that expediting is always itself directed from the beginning toward furthering something else, i.e., toward driving on to the maximum yield at the minimum expense. The coal that has been hauled out in some mining district has not been supplied in order that it may simply be present somewhere or other. It is stockpiled; that is, it is on call, ready to deliver the sun's warmth that is stored in it. The sun's warmth is challenged forth for heat, which in turn is ordered to deliver steam whose pressure turns the wheels that keep a factory running.

The hydroelectric plant is set into the current of the Rhine. It sets the Rhine to supplying its hydraulic pressure, which then sets the turbines turning. This turning sets those machines in motion whose thrust sets going the electric current for which the long-distance power station and its network of cables are set up to dispatch electricity. In the context of the interlocking processes pertaining to the orderly disposition of electrical energy, even the Rhine itself appears as something at our command. The hydroelectric plant is not built into the Rhine River as was the old wooden bridge that joined bank with bank for hundreds of years. Rather the river is dammed up into the power plant. What the river is now, namely, a water power supplier, derives from out of the essence of the power station. In order that we may even remotely consider the monstrousness that reigns here, let us ponder for a moment the contrast that speaks out of the two titles, ``The Rhine" as dammed up into the \textit{power} works, and ``The Rhine" as uttered out of the \textit{art} work, in H\"{o}lderlin's hymn by that name. But, it will be replied, the Rhine is still a river in the landscape, is it not? Perhaps. But how? In no other way than as an object on call for inspection by a tour group ordered there by the vacation industry.

The revealing that rules throughout modern technology has the character of a setting-upon, in the sense of a challenging-forth. That challenging happens in that the energy concealed in nature is unlocked, what is unlocked is transformed, what is transformed is stored up, what is stored up is, in turn, distributed and what is distributed is switched about ever anew. Unlocking, transforming, storing, distributing, and switching about are ways of revealing. But the revealing never simply comes to an end. Neither does it run off into the indeterminate. The revealing reveals to itself its own manifoldly interlocking paths, through regulating their course. This regulating itself is, for its part, everywhere secured. Regulating and securing even become the chief characterisitics of the challenging revealing.

What kind of unconcealment is it, then, that is peculiar to that which comes to stand forth through this setting-upon that challenges? Everywhere everything is ordered to stand by, to be immediately at hand, indeed to stand there just so that it may be on call for a further ordering. Whatever is ordered about in this way has its own standing. We call it the standing-reserve [\textit{Bestand}]. The word expresses here something more, and something more essential, than mere ``stock". The name ``standing-reserve" assumes the rank of an inclusive rubric. It designates nothing less than the way in which everything presences that is wrought upon by the challenging revealing. Whatever stands by in the sense of standing-reserve no longer stands over against us as object.

Yet an airliner that stands on the runway is surely an object. Certainly. We can represent the machine so. But then it conceals itself as to what and how it is. Revealed, it stands on the taxi strip only as standing-reserve, insasmuch as it is ordered to ensrue the possibility of transportation. For this it must be in its whole structure and in every one of its constituent parts, on call for duty, i.e., ready for takeoff. (Here it would be appropriate to discuss Hegel's definition of the machine as an autonomous tool. When applied to the tools of the craftsman, his characterization is correct. Characterized this way, however, the machine is not thought at all from out of the essence of technology within in which it belongs. Seen in terms of the standing-reserve, the machine is completely unautonomous, for it has its standing only from the ordering of the orderable.)

The fact that now, wherever we try to point to modern technology as the challenging revealing, the words ``setting-upon," ``ordering," ``standing-reserve," obtrude and accumulate in a dry, monotonous, and therefore oppressive way, has its basis in what is now coming to utterance.

Who accomplishes the challenging setting-upon through which what we call the real is revealed as standing-reserve? Obviously, man. To what extent is man capable of such a revealing? Man can indeed conceive, fashion, and carry through this or that in one way or another. But man does not have control over unconcealment itself, in which at any given time the real shows itself or withdraws. The fact that the real has been showing itself in the light of Ideas ever since the time of Plato, Plato did not bring about. The thinker only responded to what addressed itself to him.

Only to the extent that man for his part is already challenged to exploit the energies of nature can this ordering revealing happen. If man is challenged, ordered, to do this, then does not man himself belong even more originally than nature within the standing-reserve? The current talk about human resources, about the supply of patients for a clinic, gives evidence of this. The forester who, in the wood, measures the felled timber and to all appearances walks the same forest path in the same way as did his grandfather is today commanded by profit-making in the lumber industry, whether he knows it or not. He is made subordinate to the orderability of cellulose, which for its part is challenged forth by the need for paper, which is then delivered to newspapers and illustrated magazines. The latter, in their turn, set public opinion to swallowing what is printed, so that a set configuration of opinion becomes available on demand. Yet precisely because man is challenged more originally than are the energies of nature, i.e., into the process of ordering, he never is transformed into mere standing-reserve. Since man drives technology forward, he takes part in ordering as a way of revealing. But the unconcealment itself, within which ordering unfolds, is never a human handiwork, any more than is the realm through which man is already passing every time he as a subject relates to an object.

Where and how does this revealing happen if it is no mere handiwork of man? We need not look far. We need only apprehend in an unbiased way That which has already claimed man and has done so, so decisively that he can only b eman at any givne time as the one so claimed. Wherever man opens his eyes and ears, unlocks his heart, and gives himself over to meditating and striving, shaping and working, entreating and thinking, he finds himself everywher ealready brought into the unconcealed. The unconcealment of the unconcealed has already come to pass whenever it calls man forth into the modes of revealing allotted to him. When man, in this way, from within unconcealment reveals that which presences, he merely responds to the call of unconcealment even when he contradicts it. Thus when man, investigating, observing, ensnares nature as an area of his own conceiving, he has already been claimed by a way  of revealing that challenges him to approach nature as an object of research, until even the object disappears into the objectlessness of standing-reserve.

Modern technology as an ordering revealing is, then, no merely human doing. Therefore we must take that challenging that sets upon man to order the real as standing-reserve in accordance with the way in which it shows itself. That challenging gathers man into ordering. This gathering concentrates man upon ordering the real as standing-reserve.

That which primoridally unfolds the mountains into mountain ranges and courses through them in their folded togetherness is the gathering that we call ``\textit{Ge}birg" [mountain chain].

That original gathering from which unfold the ways in which we have feelings of one kind or another we name ``\textit{Ge}m\"{u}t" [disposition].

We now name that challenging claim which gathers man thither to order the self-revealing as standing-reserve: ``\textit{Ge-stell}" [Enframing].

We dare to use this word in a sense that has been thoroughly unfamiliar up to now.

According to ordinary usage, the word \textit{Gestell} [frame] means some kind of apparatus, e.g., a bookrack. \textit{Gestell} is also the name for a skeleton. And the employment of the word \texit{Ge-stell} [Enframing] that is now required of us seems equally eerie, not to speak of the arbitrariness with which words of a mature language are thus misused. Can anything be more strange? Surely not. Yet this strangeness is an old usage of thinking. And indeed thinkers accord with this usage precisely at the point where it is a matter of thinking that which is highest. We, late born, are no longer in a position to appreciate the significance of Plato's daring to use the word \textit{eidos} for that which in everything and in each particular thing endures as present. For \texit{eidos}, in the common speech, meant the outward aspect that a visible thing offers to the physical eye. Plato exacts of this word, however, something utterly extraordinary: that it name what precisely is not and never will be perceivable with physical eyes. But even this is by no means the full extent of what is extraordinary here. For \textit{idea} names not only the nonsensuous aspect of what is physically visible. Aspect (\textit{idea}) names and is, also, that which constitues the essence in the audible, the tasteable, the tactile, in everything that is in any way accessible. Compared with the demands that Plato makes on language and thought in this and other instances, the use of the word \textit{Gestell} as the name for the essence of modern technology, which we now venture here, is almost harmless. Even so, the usage now required remains something exacting and is open to misinterpretation.

Enframing means the gathering together of that setting-upon which sets upon man, i.e., challenges him forth, to reveal the real, in the mode of ordering, as standing-reserve. Enframing means that way of revealing which holds sway in the essence of modern technology and which is itself nothing technological. On the other hand, all those things that are so familiar to us and are standard parts of assembly, such as rods, pistons, and chassis, belong to the technological. The assembly itself, however, together with the aforementioned stockparts, falls within the sphere of technological activity; and this activity always merely responds to the challenge of Enframing, but it never comprises Enframing itself or brings it about.

The word \textit{stellen} [to set upon] in the name \textit{Ge-stell} [Enframing] not only means challenging. At the same time it should preserve the suggestion of another \textit{Stellen} from which it stems, namely, that producing and presenting [\textit{Her- und Dar-stellen}] which, in the sense of \textit{poi\-{e}sis}, lets what presences come forth into unconcealment. This producing that brings forth -- e.g., the erecting of a statue in the temple precinct -- and the challenging ordering now under consideration are indeed fundamentally different, and yet they remain related in their essence. Both are ways of revealing, of \textit{al\-{e}thia}. In Enframing, that unconcealment comes to pass in conformity with which the work of modern technology reveals the real as standing-reserve. This work is therefore neither only a human activity nor a mere means within such activity. The merely instrumental, merely anthropological definition of technology is therefore in principle untenable. And it cannot be rounded out by being referred back to some metaphysical or religious explanation that undergirds it.

It remains true, nonetheless, that man in the technological age is, in a particularly striking way, challenged forth into revealing. That revealing concerns nature, above all, as the chief storehouse of the standing energy reserve. Accordingly, man's ordering attitude and behavior display themselves first in the rise of modern physics as an exact science. Modern science's way of representing pursues and entraps nature as a calculable coherence of forces. Modern physics is not experimental physics because it applies apparatus to the questioning of nature. Rather the reverse is true. Because physics, indeed already as pure theory, sets nature up to exhibit itself as a coherence of forces calculable in advance, it therefore orders its experiments precisely for the purpose of asking whether and how nature reports itself when set up in this way.

But after all, mathematical physics arose almost two centuries before technology. How, then, could it have already been set upon by modern technology and placed in its service? The facts testify to the contrary. Surely technology got under way only when it could be supported by exact physical science. Reckoned chronologically, this is correct. Thought historically, it does not hit upon the truth.

The modern physical theory of nature prepares the way first not simply for technology but for the essence of modern technology. For already in physics the challenging gathering-together into ordering revealing holds sway. But in it that gathering does not yet come expressly to appearance. Modern physics is the herald of Enframing, a herald whose origin is still unknown. The essence of modern technology has for a long time been concealing itself, even where power machinery has been invented, where electrical technology is in full swing, and where atomic technology is well under way.

All coming to presence, not only modern technology, keeps itself everywhere concealed to the last. Nevertheless, it remains, with respect to its holding sway, that which precedes all: the earliest. The Greek thinkers already knew of this when they said: That which is earlier with regard to the arising that holds sway becomes manifest to us men only later. That which is primally early shows itself only ultimately to men. Therefore, in the realm of thinking, a painstaking effort to think through still more primally what was primally thought is not the absurd wish to revive what is past, but rather the sober readiness to be astounded before the coming of what is early.

Chronologically speaking, modern physical science begins in the seventeenth century. In contrast, machine-power technology develops only in the second half of the eighteenth century. But modern technology, which for chronological reckoning is the later, is, from the point of view of the essence holding sway within it, the historically earlier.

If modern physics must resign itself ever increasingly to the fact that its realm of representation remains inscrutable and incapable of being visualized, this resignation is not dictated by any committee of researchers. It is challenged forth by the rule of Enframing, which demands that nature be orderable as standing-reserve. Hence physics, in all its retreating from the representation turned only toward objects that has alone been standard until recently, will never be able to renounce this one thing: that nature reports itself in some way or other that is identifiable through calculation and that it remains orderable as a system of information. This system is determined, then, out of a causality that has changed once again. Causality now displays neither the character of the occasioning that brings forth nor the nature of the \textit{causa efficiens}, let alone that of the \textit{causa formalis}. It seems as though causality is shrinking into a reporting -- a reporting challenged forth -- of standing-reserves that must be guaranteed either simultaneously or in sequence. To this shrinking would correspond the process of growing resignation that Heisenberg's lecture depicts in so impressive a manner.*

*W. Heisenberg, ``Das Naturbild in der heutigen Physik," in \texit{Die K\"{u}nste im technischen Zeitaler} (Munich, 1954), pp. 43 ff.

Because the essence of modern technology lies in Enframing, modern technology must employ exact physical science. Through its so doing, the deceptive illusion arises that modern technology is applied physical science. This illusion can maintain itself only so long as neither the essential origin of modern science nor indeed the essence of modern technology is adequately found out through questioning.

We are questioning concerning technology in order to bring to light our relationship to its essence. The essence of modern technology shows itself in what we call Enframing. But simply to point to this is still no way to answer the question concerning technology, if to answer means to respond, in the sense of correspond, to the essence of what is being asked about.

Where do we find ourselves brought to, if now we think one step further regarding what Enframing itself actually is? It is nothing technological, nothing on the order of a machine. It is the way in which the real reveals itself as standing-reserve. Again, we ask: Does this revealing happen somewhere beyond all human doing? No. But neither does it happen exclusively \textit{in} man, or decisively \textit{through} man.

Enframing is the gathering together that belongs to that setting-upon which sets upon man and puts him in position to reveal the real, in the mode of ordering, as standing-reserve. As the one who is challenged forth in this way, man stands within the essential realm of Enframing. He can never take up a relationship to it only subsequently. Thus the question as to how we are to arrive at a relationship to the essence of technology, asked in this way, always comes too late. But never too late comes the question as to whether we actually experience ourselves as the ones whose activities are everywhere, public and private, are challenged forth by Enframing. Above all, never too late comes to question as to whether and how we actually admit ourselves into that wherein Enframing itself comes to presence.

The essence of modern technology starts man upon the way of that revealing through which the real everywhere, more or less distinctly, becomes standing-reserve. ``To start upon a way" means ``to send" in our ordinary language. We shall call that sending-that-gathers [\textit{versammele Schicken}] which first starts man upon a way of revealing, \textit{destining} [\textit{Geschick}]. It is from out of this destining that the essence of all history [\textit{Geschichte}] is determined. History is neither simply the object of written chronicle nor simply the fulfillment of human activity. That activity first becomes history as something destined. And it is only the destining into objectifying representation that makes the historical accessible as an object for historiography, i.e., for a science, and on this basis makes possible the current equating of the historical with that which is chronicled.

Enframing, as a challenging-forth into ordering, sends into a way of revealing. Enframing is an ordaining of destining, as is every way of revealing. Bringing-forth, \texit{po\-{e}sis}, is also a destining in this sense. Always the unconcealment of that which is goes upon a way of revealing. Always the destining of revealing holds complete sway over man. But that destining is never a fate that compels. For man becomes truly free only insofar as he belongs to the realm of destining and so becomes one who listens and hears [\textit{H\"{o}render}], and not one who is simply constrained to obey [\textit{H\"{o}riger}].

The essence of freedom is \textit{originally} not connected with the will or even with the causality of human willing.

Freedom governs the open in the sense of the cleared and lighted up, i.e., of the revealed. It is to the happening of revealing, i.e., of truth, that freedom stands in the closest and most intimate kinship. All revealing belongs within a harboring and a concealing. But that which frees -- the mystery -- is concealed and always concealing itself. All revealing comes out of the open, goes into the open, and brings into the open. The freedom of the open consists neither in unfettered arbitrariness nor in the constraint of mere laws. Freedom is that which conceals in a way that opens to light, in whose clearing there shimmers that veil that covers what comes to presence of all truth and lets the veil appear as what veils. Freedom is the realm of the destining that at any given time starts a revealing upon its way.

The essence of modern technology lies in Enframing. Enframing belongs within the destining of revealing. These sentences express something different from the talk that we hear more frequently, to the effect that technology is the fate of our age, where ``fate" means the inevitableness of an unalterable course.

But when we consider the essence of technology, then we experience Enframing as a destining of revealing. In this way we are already sojourning within the open space of destining, a destining that in no way confines us to a stultified compulsion to push on blindly with technology or, what comes to the same thing, to rebel helplessly against it and curse it as the work of the devil. Quite to the contrary, when we once open ourselves expressly to the \textit{essence} of technology, we find ourselves unexpectedly taken into a freeing claim.

The essence of technology lies in Enframing. Its holding sway belongs with destining. Since destining at any given time starts man on a way of revealing, man, thus under way, is continually approaching the brink of the possibility of pursuing and pushing forward nothing but what is revealed in ordering, and of deriving all his standards on this basis. Through this the other possibility is blocked, that man might be admitted more and sooner and ever more primally to the essence of that which is unconcealed and to its unconcealment, in order that he might experience as his essence his needed belonging to revealing.

Placed between these possibilities, man is endangered from out of destining. The destining of revealing is as such, in every one of its modes, and therefore necessarily, \texit{danger}.

In whatever way the destining of revealing may hold sway, the unconcealment in which everything that is shows itself at any given time harbors the danger that man may quail at the unconcealed and may misinterpret it. Thus where everything that presences exhibits itself in the light of a cause-effect coherence, even God can, for representational thinking, lose all that is exalted and holy, the mysteriousness of his distance. In the light of causality, God can sink to the level of a cause, of \textit{causa efficiens}. He then becomes, even in theology, the god of the philosophers, namely, of those who define the unconcealed and the concealed in terms of the causality of making, without ever considering the essential origin of this causality.

In a similar way the unconcealment in accordance with which nature presents itself as a calculable complex of the effects of forces can indeed permit correct determinations; but precisely through these successes the danger can remain that in the midst of all that is correct the true will withdraw.

The destining of revealing is in itself not just any danger, but danger as such.

Yet when destining reigns in the mode of Enframing, it is the supreme danger. This danger attests itself to us in two ways. As soon as what is unconcealed no longer concerns man even as object, but does so, rather, exclusively as standing-reserve, and man in the midst of objectlessness is nothing but the orderer of the standing-reserve, then he comes to the very brink of a precipitous fall; that is, he comes to the point where he himself will have to be taken as standing-reserve. Meanwhile man, precisely as the one so threatened, exalts himself to the posture of lord of the earth. In this way the impression comes to prevail that everything man encounters exists only insofar as it is his construct. This illusion gives rise in turn to one final delusion: It seems as though man everywhere and always encounters only himself. Heisenberg has with complete correctness pointed out that the real must present itself to contemporary man in this way.* \texit{In truth, however, precisely nowhere does man today any longer encounter himself, i.e., his essence}. Man stands so decisively in attendance on the challenging-forth of Enframing that he does not apprehend Enframing as a claim, that he fails to see himself as the one spoken to, and hence also fails in every way to hear in what respect he ek-sists, from out of his essence, in the realm of an exhortation or address, and thus \textit{can never} encounter only himself.

But Enframing does not simply endanger man in his relationship to himself and to everything that is. As a destining, it banishes man into that kind of revealing which is ordering. Where this ordering holds sway, it drives out every other possibility of revealing. Above all, Enframing conceals that revealing which, in the sense of \textit{p\-{o}esis}, lets what presences come forth into appearance. As compared with that other revealing, the setting-upon that challenges forth thrusts man into a relation to that which is, that is at once antithetical and rigorously ordered. Where Enframing holds sway, regulating and securing of the standing-reserve mark all revealing. They no longer even let their own fundamental characteristic appear, namely, this revealing as such.

Thus the challenging Enframing not only conceals a former way of revealing, bringing-forth, but it conceals revealing itself and with it That wherein unconcealment, i.e., truth, comes to pass.

Enframing blocks the shining-forth and holding-sway of truth. The destining that sends into ordering is consequently the extreme danger. What is dangerous is not technology. There is no demonry in technology, but rather there is the mystery of its essence. The essence of technology, as a destining of revealing, is the danger. The transformed meaning of the word ``Enframing" will perhaps become somewhat more familiar to us now if we think Enframing in the sense of destining and danger.

The threat to man does not come in the first instance from the potentially lethal machines and apparatus of technology. The actual threat has already affected man in his essence. The rule of Enframing threatens man with the possibility that it could be denied to him to enter into a more original revealing and hence to experience the call of a more primal truth.

Thus, where Enframing reigns, there is \textit{danger} in the highest sense.

> \textit{But where danger is, grows}
> \textit{The saving power also}

Let us think carefully about these words of H\"{o}lderlin. What does it mean ``to save"? Usually we think that it means only to seize hold of a thing threatened by ruin, in order to secure it in its former continuance. But the verb ``to save" says more. ``To save" is to fetch something home into its essence, in order to bring the essence for the first time into its genuine appearing. If the essence of technology, Enframing, is the extreme danger, and if there is no truth in H\"{o}lderlin's words, then the rule of Enframing cannot exhaust itself solely in blocking all lighting-up of every revealing, all appearing of truth. Rather, precisely the essence of technology must harbor in itself the growth of the saving power. But in that case, might not an adequate look into what Enframing is as a destining of revealing bring into appearance the saving power in its arising? 

In what respect does the saving power grow there also where the danger is? Where something grows, there it takes root, from thence it thrives. Both happen concealedly and quietly and in their own time. But according to the words of the poet we have no right whatsoever to expect that there where the danger is we should be able to lay hold of the saving power immediately and without preparation. Therefore we must consider now, in advance, in what respect the saving power does most profoundly take root and thence thrive even in that wherein the exteme danger lies, in the holding sway of Enframing. In order to consider this, it is necessary, as a last step on our way, to look with yet clearer eyes into the danger. Accordingly, we must once more question concerning technology. For we have said that in technology's essence roots and thrives the saving power.
