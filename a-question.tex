\documentclass[12pt]{article}
\usepackage{lingmacros}

\title{The Question Concerning Technology}
\author{Martin Heidegger \thanks{William Lovitt}}
\date{April 2019}
\begin{document}

\maketitle

In what follows we shall be \textit{questioning} concerning technology. Questioning builds a way. We would be advised, therefore, above all to pay heed to the way, and not to fix our attention on isolated sentences and topics. The way is a way of thinking. All ways of thinking (ahaha), more or less perceptibly, lead through language in a manner that is extraoridnary. We shall be questioning concerning \textit{technology}, and in doing so we should like to prepare a free relationship to it. The relationship will be free if it opens our human existence to the essence of technology. When we can respond to this essence, we shall be able to experience the technological within its own bounds.

Technology is not equivalent to the essence of technology. When we are seeking the essence of ``tree", we have to become aware that That which pervades every tree, as tree, is not itself a tree that can be encountered among all the other trees.

Likewise, the essence of technology is by no means anything technological. Thus we shall never experience our relationship to the essence of technology so long as we merely conceive and push forward the technological, put up with it, or evade it. Everywhere we remain unfree and chained to technology, whether we passionately affirm or deny it. But we are delivered over to it in the worst way possible when we regard it as something neutral; for this conception of it, to which today we are particularly like to do homage, makes us utterly blind to the essence of technology.

According to ancient doctrine, the essence of thing is considered to be \textit{what} the thing is. We ask the question concerning technology when we ask what it is. Everyone knows the two statements that answe rour question. One says: Technology is a means to an end. The other says: Technology is a human activity. The two definitions of technology belong together. For to posit ends and procure and utilize the means to them is a human activity. The manufacture and utilization of equipment, tools, and machines, the manufactured and used things themselves, and the needs and ends that they serve -- all belong to what technology is. The whole complex of these contrivances is technology. Technology itself is a contrivance, or, in latin, an \textit{instrumentum}.

The current conception of technology, according to which it is a means and human activity, can therefore be called the instrumental and anthropological definition of technology.

Who would ever call this incorrect? It obviously conforms with what we envision when we talk about technology. The instrumental definition of technology is in fact so uncannily correct that it even holds for modern technology, of which, in other respects, we maintain with some justification that it is, in contrast to the older handwork technology, something completely different and therefore new. Even the power plant with its turbines and generators is a man-made means to an end established by man. Even the jet aircraft and the high-frequency radio are means to ends. A radar station is of course less simple than a weather vane. To be sure, the construction of a high-frequency device requires the interlocking of various processes of technical-industrial production. And certainly a sawmill in a secluded valley of the Black Forest is a primitive means compared with the hydroelectric plant in the Rhine River.
the distinctive expression of the cognitive or intellectual character of a culture or social group.

But this much remains correct: modern technology too is a means to an end. That is why the instrumental conception of technology conditions every attempt to bring man into the right relation to technology. Everything depends on our manipulating technology in the rightly as a means. We will, as we say, "get" technology "spirtutally in hand". We will master it. The will to mastery becomes all the more urgent the more technology threatens to slip from human control.

Supposing that technology is not merely means, how does that square with the will to master it? Did we not say that the definition of technology as an instrument is correct? Certainly. The correct always fixes upon something suitable in whatever is under consideration. However, in order to be correct, this 'fixing' needs not to uncover the essence of the thing in question. Only at the point when the essential uncovering happens does the 'true' become known. This is why the merely correct is not yet also the 'true'. Only the true brings us into a free relationship with that of its (technology's) essence that concerns us. Thus, the correct technical definition of technology does not show us technology's essence. So that we may arrive at this essence, or at least come close to it, we will find the true by way of the correct. We must ask: What is the technical itself? Within what do such things as 'means' and 'ends' belong? A means is where something is precipitated and hence attained. Whatever has an 'effect' as its consequence is called a cause. But not just the means that cause 'effect' are a cause. The end, by lending to the determination of which \textit{type} of means will be used, is also considered a cause. Wherever ends are pursued and means employed, wherever technicality reigns, there reigns also causality.

For centuries philosophy has taught that there are four causes: (1) the \textit{causa materialis} (material cause), the matter from which, for example, a silver chalice is made; (2) the \textit{cause formalis} (formal cause), the form, the shape the material forms; (3) the \textit{causa finalis} (final cause), the end or goal, for example, the relationship between the chalice and sacrificial rite which determines the necessary form and material for it; (4) the \textit{causa efficiens} (efficient cause), which brings about the \textit{effect} that is the finished, actual chalice -- in this case, the silversmith. What technology is, understood as means, reveals itself when we trace technicality back to the four 'causalities'.

Should we suppose that causality, for its part, is veiled in darkness as to what it actually is? No doubt, for centuries we have acted as though the doctrine of the four causes had fallen from heaven, a truth as clear as day. But maybe the time has come to ask: Why only four causes? In context of the aforementioned four, what does "cause" really mean? How did we arrive at the singular determination that the \textit{character} of the four causes is such that they belong together?

As long as we avoid these questions, causality, and with it technicality, and with technicality the accepted definition of technology, remain obscure and groundless.

We are long accustomed to understanding 'cause' as that which brings something else about. In this connotation, to 'bring something about' means to obtain results or effects. The \textit{causa efficiens}, just one among the four causes, thus sets the standard for our understanding of 'causality'. This interpretation carries so far that we no longer even count the \textit{causa finalis}, end finality, as causality. \textit{Causa}, \textit{casus}, belongs to the verb \textit{cadere}, ``to fall," and can be defined as ``that which brings about the result, that something falls out in such and such a way". The doctrine of the four causes goes back to Aristotle. But everything that later ages look to find in Greek thought under the conception and rubric ``causality," as Greek thought per se had nothing at all to do with bringing about and effecting. What we call cause [\textit{Ursache}] and the Romans call /{causa} is called \textit{aition} by the Greeks, which meant ``that to which something else is indebted". The four causes are the method, indebted simultaneously to each other, of being responsible for something else. An example will help here.

The silver chalice is made out of silver. As this substance (\textit{hyle}), it is co-responsible for the chalice. The chalice is indebted to, i.e., owes thanks to, the silver that it is made from. But the sacrificial vessel, the chalice, is indebted not only to the silver material. As a chalice, that which is indebted to the silver appears in the shape of a chalice, not as a brooch or a ring. The sacrificial vessel is simultaneously indebted to the idea (\textit{eidos}) of chaliceness. Both the silver into which the shape is formed as chalice and the shape in which the silver appears are in their respective ways co-responsible for the sacrificial vessel.

There is yet a third participant that has utmost responsiblity for the sacrificial vessel. This participant is that which restricts the chalice to the domain of consecration and bestowal. Through this the chalice is defined as sacrificial vessel. Definition gives bounds to the object. These bounds do not stop the object; rather from out of them it begins to be what, after production, it will be. That which gives bounds, that which completes, in this sense is called in Greek \textit{telos} -- all too often traslated as ``aim" or ``purpose," and so misinterpreted. The \textit{telos} is responsible for what as matter and for what as shape are together co-responsible for the sacrificial vessel.

Finally there is a fourth participant in the responsibility for the finished sacrificial vessel lying before us ready for use, i.e., the silversmith -- but not because he, in working, brings about the finished sacrficial chalice, as if it were the effect of a making; the silversmith is not a \textit{causa efficiens}.

The Aristolian doctrine neither knows the cause that is named by this term nor uses a Greek word that would correspond to it.

The silversmith carefully considers and brings together the other three ways of being responsible and indebted. To carefully consider is in Greek \textit{legein}, \textit{logos}. \textit{Legein} is rooted in \textit{apophainesthai}, to bring forward into appearance. The silversmith is co-responsible as that from which the sacrificial vessel's 'bringing forth' and being-in-self take and keep their first deviation. The three previously mentioned means of responsiblity are indebted to the ponderous consideration of the silversmith for the ``that" and the ``how" of their coming to appear and to play in the production of the sacrificial vessel.

Thus these four responsiblities predominate in the sacrifical vessel that lies finished before us. They differ from each other, yet belong together. What unites them from the beginning? In what does this unified performance of the four responsibilities play? What is the source of the unity of the four causes? What, after all, does this debt and responsibility mean, viewed as the Greeks thought it? 

Today we are too willing to either understand responsiblity and indebtedness as a moral lapse, or else interpret them in terms of effecting. In either case, we bar ourselves from the way to the primal meaning of that which is to be called causality. So long as this way is closed to us we shall also fail to see what technicality, which is based on causality, actually is.

In order to guard against such misinterpretations of responsiblity and indebtedness, let us clarify the four responsiblities in terms of that for which they are responsible. According to our example, they are together responsible for the silver chalice lying finished before us as a sacrificial vessel. Lying before and lying finished (\textit{hypokeisthai}) characterize the 'manifestation' of something that manifests. The four responsiblities, together, bring something into being. They let it come forth into 'manifestation'. They set it free and start it on its way, namely, into its complete arrival. The principal characteristic of responsiblity is this 'starting something on its way into arrival'. It is in the sense of such a 'starting something on its way to arrival' that responsiblity is an occasioning, an inducement to go forward. Based on observation of what the Greeks experienced with regards to responsiblity, in \textit{aitia}, we can give the verb ``to occasion" a more inclusive meaning, thus it now is the name for the essence of causality viewed as the Greeks thought it. The common and narrower meaning of ``occasion" in contrast is nothing more than striking against and releasing, and means a kind of secondary cause within the whole causality.

But in what, then, does the unified performance of the four ways of occasioning play? These ways let what is not yet present arrive into manifestation. As such, they are singularly governed by a \texit{bringing} that brings what manifests into being. Plato tells us what this bringing is in this sentence of the \textit{Symposium} (205b): \textit{h\-{e} gar toi ek tou m\-{e} onton eis to on ionti hot\-{o}ioun aitia pasa esti poi\-{e}sis}. ``Every occasion for whatever passes over and goes on into manifestation from that which is not manifested is \textit{poi\-{e}sis}, is bringing-forth."

It is of utmost importance that we understand bringing-forth both in its full scope and also in the sense that the Greeks understood it. Not just handcraft manufacture, not just artistic and poetical bringing into being and concrete imagery, is a bringing-forth, \textit{poi\-{e}sis}. \testit{Physis} also, the arising of something from out of itself, is a bringing-forth, \textit{poi\-{e}sis}. \textit{Physis} is indeed \textit{poi\-{e}sis} in the highest sense. For what manifests by means of \textit{physis} has the same bursting open of bringing-forth, e.g., the bursting of a blossom into bloom, in itself (\textit{en heaut\-{o}i}). In contrast, what is brought forth by the artisan or the artist, e.g., the silver chalice, has the bursting open of bringing-forth not in itself, but in another (\textit{en all\-{o}i}), in the craftsman or artist.

The modes of occasioning, the four causes, are at play, then, in bringing-forth. Through bringing-forth, the growing things of nature as well as whatever is finished through arts and crafts arrive to their being.

But how does bringing-forth happen, in nature or in handwork or in art? What is this bringing-forth that the fourfold way of 'occasioning' plays within? 'Occasioning' concerns the manifestation of that which comes to being in bringing-forth. Bringing-forth brings hither out of concealment, forth to un-concealment. Bringing-forth comes to pass only insofar as something concealed comes to be un-concealed. This coming exists and moves freely within what we call revealing. The Greeks have the word \textit{al\-{e}theia} for revealing. The Romans translate this with \textit{veritas}. We say ``truth" and usually understand it as the correctness of an idea.



Yet where have we strayed to? We started with questions concerning technology, and now we have arrived at \textit{al\-{e}theia}, at revealing. What does the essence of technology have to do with revealing? The answer: everything. Because every bringing-forth is rooted in revealing. Bringing-forth, indeed, gathers within itself the four modes of occasioning -- causality -- and governs them throughout. Within its (bringing-forth's) domain belong end and means, belongs technicality. Technicality is considered to be the fundamental characteristic of technology. If we inquire, step by step, into what technology, represented as means, actually is, then we will arrive at revealing. The possibility of all productive manufacturing lies in revealing.

Technology is no mere means. Technology is a way of revealing. If we pay attention to this, then another entire domain for the essence of technology will open itself to us. It is the domain of revealing, i.e., of truth.

This possibility strikes us as strange. Indeed, it should do so, should do so as persistently as possible and with so much urgency that we will finally take seriously the simple question of what the name ``technology" means. The word stems from the Greek. \textit{Technikon} means that which belongs to \textit{techn\-{e}}. We should note two things about the meaning of this word. One is that \textit{techn\-{e}} is the name not only for the activities and skills of the craftsman, but also for the arts of the mind and the fine arts. \textit{Techn\-{e}} belongs to bringing-forth, to \{poi\-{e}sis}; it is something poietic.

The other point that we should note about \textit{techn\-{e}} is even more important. From the earliest times up until Plato the word \textit{techn\-{e}} is linked with the word \textit{epist\-{e}m\-{e}}. Both words are names for knowing in the widest sense. They mean to be entirely at home in something, to understand and be an expert in it. Such knowing provides an opening up. As an opening up it is a revealing. Aristotle, in a discussion of special importance (\textit{Nichomachean Ethics}, Bk. VI, chaps. 3 and 4), distinguishes between \textit{epist\-{e}m\-{e}} and \textit{techn\-{e}}, with respect to what and how they reveal. \textit{Techn\-{e}} is a mode of \textit{al\-{e}theuein}. It reveals whatever does not bring itself forth and does not yet exist here before us, whatever can look and turn out now one way and now another. Whoever builds a house or a ship or forges a sacrificial chalice reveals what is to be brought forth, according to the four ways of 'occasioning'. This revealing gathers together in advance the form and the material of ship or house, with a vision of the thing completed, and from this gathering determines the manner of its construction. What is decisive in \textit{techn\-{e}} does not lie at all in making and manipulating nor in the using of means, but rather in the aforementioned revealing. It is as revealing, and not as manufacturing, that \textit{techn\-{e}} is a bringing-forth.

The clue then to what \textit{techn\-{e}} means and to how the Greeks defined it leads us to the same context that opened to us when we pursued the question of what technicality, as such, truthfully, might be.

Technology is a way of revealing. Technology manifests in the domain where revealing and un-concealment take place, where \textit{al\-{e}thia}, truth, happens.

One can object to this definition of the essential domain of technology that it indeed holds for Greek thought and that at best it might apply to the techniques of the handcraftsman, but that it simply does not fit modern machine-powered technology. It is precisely this objection, and it alone, that is the distrubing thing, that moves us to ask the question concerning technology per se. It is said that modern technology is something incomparably different from all earlier technologies because it is based on modern physics as an exact science. Meanwhile, we have come to understand more clearly that the reverse holds true as well: Modern physics, as experimental, is dependent upon technical apparatus and upon progress in the building of apparatus. The establishment of this mutual relationship between technology and physics is correct. But it remains a merely historiographical establishment of facts and says nothing about what this mutual relationship is grounded in. The decisive question still remains: what is the essence of modern technology such that it happens to think of putting exact science to use?

What is modern technology? It too is a revealing. Only when we rest our attention on this fundamental characteristic does what is new in modern technology reveal itself to us. 

And yet the revealing that predominates throughout modern technology does not unfold into a bringing-forth in the sense of \textit{po\-{e}sis}. The revealing that dominates in modern technology is a \texit{challenging}, which unreasonably demands from nature a supply of energy that can be extracted and stored as such. But is this not true for the old windmill as well? No. The windmill's sails do indeed turn in the wind; in fact they are left entirely to the wind's blowing. But the windmill does not unlock energy from the air currents in order to store it.

In contrast, a tract of land is exacted into putting out coal and ore. The earth now reveals itself as a coal mining district, the soil as a mineral desposit. The field that the peasant formerly cultivated and settled appears differently than it did when 'to settle' still meant to take care of and to maintain. The work of the peasant does not exact the soil of the field. By sowing the grain, the peasant places the seed into the care of the forces of growth and watches over its development. But meanwhile even the cultivation of the field has incurred another kind of 'settling', which \textit{settles} upon nature. It settles upon it in the sense of exacting it. Agriculture is now the mechanized food industry. Air is now settled upon to yield nitrogen, the earth to yield ore, ore to yield uranium, for example; uranium is settled to yield atomic energy, which can be released either for destruction or for peaceful use.

This settling that exacts the energies of nature is an acceleration, in two ways. It accelerates in that it unlocks and exposes. Yet that acceleration is itself always directed, from the outset, toward furthering something else, i.e., toward driving out the maximum yield at the minimum expense. The coal that has been hauled out in some mining district has not been supplied so that it may simply be \textit{present} somewhere or other. It is stockpiled; that is, it is on call, ready to deliver the sun's warmth that is stored in it. The sun's warmth is exacted for heat, which in turn is ordered to deliver steam whose pressure turns the wheels that keep a factory running.

The hydroelectric plant is settled into the Rhine river's current. It sets the Rhine to supplying hydraulic pressure, which then sets the turbines turning. This turning sets machines in motion, machines whose thrust gets going the electric current for which the long-distance power relay station and its network of cables are set up to dispatch electricity. Viewed in this context of interlocking processes stemming from the orderly dispersal of electrical energy, even the Rhine itself appears as something at our command. The hydroelectric plant is not built into the Rhine River the same way as was the old wooden bridge that joined bank with bank for hundreds of years. Rather, the river is dammed up into the power plant. What the river is now, namely, a water power supplier, is derived from the essence of the power station. So that we may understand the monstrousness that reigns here, let us consider for a moment the contrast between the two titles, ``The Rhine" as dammed up into \textit{power} works, and ``The Rhine" as uttered out of the \textit{art} work, in H\"{o}lderlin's hymn by that name. But, it will be replied, the Rhine is still a river in the landscape, is it not? Perhaps. But how? In no other way than as an object on call for inspection by a tour group ordered there by the vacation industry.

The revealing that governs throughout modern technology has the character of a settling, in the sense of an exaction. That exaction happens in that the energy concealed in nature is unlocked, what is unlocked is transformed, what is transformed is stored up, what is stored up is, in turn, distributed and what is distributed is switched about ever anew. Unlocking, transforming, storing, distributing, and switching about are ways of revealing. But the revealing never simply comes to an end. Nor does it run off into the indeterminate. The revealing reveals to itself its own manifoldly interlocking paths, through regulating the interlocking paths' course. This regulating itself is, for its part, everywhere attached. Regulating and attaching become the chief characterisitics of the exacting-revealing.

What kind of un-concealment is it, then, that is particular to what comes to exist through this settling that exacts? Anywhere anything is set to stand by, to be immediately at hand, indeed to stand there just so that it may be on call for a further settling. Whatever is set about in this way has its own standing. We call it the standing-reserve. The word ``standing-reserve" expresses here something more, something more essential, than mere ``stock". The name ``standing-reserve" assumes the role of an inclusive rubric. It designates no less than how all things manifest when they are wrought upon by the exacting-revealing. Whatever stands by in the sense of standing-reserve no longer stands over against us as object.

Yet an airliner that stands on the runway is surely an object. Certainly. We can represent the machine as such. But this representation conceals as to what and how the airliner is. Once revealed, it stands on the taxi strip just as standing-reserve, insofar as it is set to ensure the possibility of transportation. For this it must be, in its whole structure and in every one of its constituent parts, on call for duty, i.e., ready for takeoff. (Here it would be appropriate to discuss Hegel's definition of the machine as an autonomous tool. When applied to the tools of the craftsman, his characterization is correct. Characterized this way, however, the machine is not considered at all outside of the essence of technology that it belongs within. Seen in terms of the standing-reserve, the machine is completely unautonomous, for it has its standing only from the setting of the set-able.)

The fact that now, wherever we try to point to modern technology as the exacting revealing, the words ``settling," ``setting," ``standing-reserve," interpose and accumulate in a dry, monotonous, and therefore oppressive, way, has its basis in what is now arriving at articulation.

Who performs the exacting setting-upon through which what we call the 'real' is revealed as standing-reserve? Obviously, man. In what capacity is man capable of such a revealing? Man can indeed conceive, fashion, and carry through notions in one way or another. But man does not have control over un-concealment itself, where at any time the real may show itself or withdraw. The fact that the real shows itself in the light of Ideas, since the time of Plato, Plato did not bring about. The thinker merely responded to what spoke to him.

Only to the extent that man, for his part, is already called to exploit the energies of nature can this settling revealing happen. If man is exacted, set, to do this, then does not man himself belong, even more primitively than nature, within the standing-reserve? The current talk about human resources, about the supply of patients for a clinic, is evidence of this. The forester who, in the forest, measures the felled trees and presumably walks the same forest path as did his grandfather is today tasked by profit-making in the lumber industry, whether he knows it or not. He is subordinated to the settle-ability of cellulose, which for its part is exacted by the need for paper, which is then delivered to newspapers and magazines. The latter, in turn, set public opinion to swallowing what is printed, such that the configuration of opinion becomes available on demand. Yet precisely because man is exacted more primitively than are the energies of nature, i.e., into the process of settling, he is not transformed into standing-reserve. Since man drives technology forward, he participates in settling as a way of revealing. But the un-concealment itself, within which settling develops, is not human handiwork, no more than is the domain through which man is already passing every time he as a subject relates to an object.

Where and how does this revealing happen if it is not the handiwork of man? We needn't look far. We only need to understand, dispassionately, What has already laid claim to man has done so, so decisively that he can only be man \textit{as} the one so claimed. Wherever man opens his eyes and ears, unlocks his heart, and gives himself over to meditating and striving, shaping and working, entreating and thinking, he finds himself everywhere already arrived at the un-concealed. The un-concealment of the un-concealed has already occured whenever it brings man to the modes of revealing granted to him. When man, thus, from within un-concealment reveals what manifests, he is merely responding to the call of un-concealment even when he contradicts it. Thus when man, investigating, observing, ensnares nature as a domain of his own conception, he has already been claimed by a way of revealing that exacts him to approach even nature itself as an object of research, until even the object disappears into the objectlessness of standing-reserve.

Modern technology as a settling-revealing is, then, no merely human doing. We must understand the exacting that sets upon man to settle the real as standing-reserve as it shows itself to be. That exacting gathers man into settling. This gathering focuses man upon settling the real as standing-reserve.

What primoridally develops the mountains into mountain ranges and runs through them in their folded togetherness is the development that we call ``\textit{Ge}birg" [mountain chain].

That original development from which develop the manner that we have one kind of feelings or another we name ``\textit{Ge}m\"{u}t" [disposition].

We now name that exacting claim which gathers man toward settling the self-revealing as standing-reserve: ``\textit{Ge-stell}" [Enframing].

We dare to use this word in a sense that has been thoroughly unfamiliar up to now.

In normal usage, the word \textit{Gestell} [frame] means some kind of apparatus, e.g., a bookshelf. \textit{Gestell} is also the name for a skeleton. The employment of the word \texit{Ge-stell} [Enframing] that is now required of us seems equally eerie, say nothing of the arbitrariness with which words of a mature language are thus misused. Can anything be more strange? Surely not. Yet this strangeness results from an old habit of thinking. And indeed thinkers agree with this usage exactly at the point where thinking is the matter at hand.  We, late born, are no longer in a position to appreciate the significance of Plato's daring to use the word \textit{eidos} for what endures, present, in each and every particular thing. For \texit{eidos}, in the common speech of the time, meant the outward appearance that a visible thing offers to the physical eye. Plato exacts of this word, however, something utterly extraordinary: that it name precisely what is not, nor ever will be, perceivable with physical eyes. This is in no way the full extent of what is extraordinary here. For \textit{idea} names not only the nonsensuous aspect of what is physically visible. Aspect (\textit{idea}) names and is, also, what constitues the essence in the audible, the tasteable, the tactile, in everything that is in any way accessible. Compared with the demands that Plato makes on language and thought, in this and other instances, the use of the word \textit{Gestell} as the name for the essence of modern technology, which we now endeavor here, is almost harmless. Even so, the usage now required remains something exacting and is open to misinterpretation.

Enframing is the coming together of the setting-upon that sets itself upon man, i.e., exacts him, to reveal the real by way of settling, as standing-reserve. Enframing is the method of revealing that predominates the essence of modern technology. Itself, it is nothing technological. However, all the things that are familiar to us and are standard parts of manufacturing -- such as rods, pistons, and chassis -- belong to the technological. Assembly itself, however, together with the aforementioned stockparts, falls within the domain of technological activity; this activity merely responds to the exacting of Enframing -- it never constitues Enframing itself or brings Enframing about.

The word \textit{stellen} [to set upon] in the phrase \textit{Ge-stell} [Enframing] not only means exacting. We should also maintain the connotation of another \textit{Stellen} that it's derived from, namely, producing and presenting [\textit{Her- und Dar-stellen}] which, in the sense of \textit{poi\-{e}sis}, lets what manifests come forth into un-concealment. This producing that delivers -- e.g., erecting a statue in the temple precinct -- and the 'exacting ordering' now under consideration are in fact fundamentally different, and yet they remain related in their essence. Both are methods of revealing, of \textit{al\-{e}thia}. In Enframing, un-concealment happens in conformity with how modern technology's work reveals the real as standing-reserve. This work is not only a human activity nor is it a mere means of such human activity. Thus the merely instrumental, merely anthropological definition of technology is in principle untenable. It cannot be rounded out by reference back to some metaphysical or religious explanation that undergirds it.

It remains true, nonetheless, that man in the technological age is, in a particularly striking way, exacted forth into revealing. That revealing involves nature, above all, as the prime warehouse of the standing energy reserve. Accordingly, man's ordering attitude and behavior show themselves first in the emergence of modern physics as an exact science. Modern science's manner of representation pursues and ensnares nature as a calculable unity of forces. Modern physics is not experimental physics because it applies machinery to the questioning of nature. Instead the reverse is true. Because physics, indeed already as pure theory, frames nature such that it exhibits itself as a calculable unity of forces, it therefore orders its experiments precisely for the purpose of asking whether and how nature reports itself when approached in this way.

Nevertheless, mathematical physics appeared almost two centuries before technology. How, then, can it have already been assailed by modern technology and placed in its service? The facts argue the contrary. Surely technology started only when it could be supported by exact physical science. Reckoned chronologically, this is correct. Thought historically, it misses the truth.

The modern physical theory of nature paves the way not simply for technology but for the essence of modern technology. For already in physics the exacting gathering-together into ordering revealing predominates. However, in physics, that gathering does not yet explicitly appear. Modern physics is the herald of Enframing, a herald whose origin is still unknown. The essence of modern technology has for long concealed itself, even where power machinery has been invented, where electrical technology is in full swing, and where atomic technology is well under way.

All manifestation, not only modern technology, keeps itself concealed to the last. Nevertheless, it remains, with respect to its predominance, what precedes all: the original. The Greek thinkers already knew of this when they said: What is original regarding the 'arising that predominates' becomes manifest to us men only later. What is original shows itself to men only at the end. Therefore, in the realm of thinking, a painstaking effort to think through still more primally what was originally thought is not the absurd wish to revive what is past, but rather the sober readiness to be astounded before the coming of what is original.

Chronologically speaking, modern physical science begins in the seventeenth century. In contrast, machine-powered technology developed only in the second half of the eighteenth century. But modern technology, which by chronological reckoning is later, is, from the point of view of the essence governing it, historically earlier.

If modern physics must increasingly resign itself to the fact that its domain of representation remains inscrutable and incapable of visualization, it is a resignation not dictated by any committee of researchers. It is exacted out by the rule of Enframing, that demands nature be orderable as standing-reserve. Hence physics, in all its retreating from the object oriented representation that has been the sole standard until recently, will never be able to renounce this: that nature reports itself in a manner that is identifiable by calculation and that it remains orderable as a system of information. This system is determined, then, out of a causality that has shifted once again. Causality now displays neither the character of the delivering event nor the nature of the \textit{causa efficiens}, let alone that of the \textit{causa formalis}. It seems as though causality is shrinking into a reporting -- a reporting exacted out -- of standing-reserves that must be guaranteed either simultaneously or in sequence.  This shrinking causality corresponds to the case of growing resignation that Heisenberg's lecture so impressively depicts.

*W. Heisenberg, ``Das Naturbild in der heutigen Physik," in \texit{Die K\"{u}nste im technischen Zeitaler} (Munich, 1954), pp. 43 ff.

Because the essence of modern technology lies in Enframing, modern technology must make use of exact physical science. Through its so doing, the deception arises that modern technology \textit{is} applied physical science. This illusion maintains itself only so long as neither the essential origin of modern science nor indeed the essence of modern technology is adequately discovered through questioning.

We are questioning concerning technology in order to uncover our relationship to technology's essence. It shows itself in what we call Enframing. But to simply point this out still is in no way an answer to the question concerning technology, if to answer means to respond, in the sense of correspond, to the essence of what is being asked about.

Where have we arrived at now, if we think one step further along, about what Enframing itself actually is? What we arrive at is nothing technological, nothing resembling a machine. Instead, we find ourselves at the manner that the real reveals itself as standing-reserve. Again, we ask: Does this revealing happen somewhere beyond human doing? No. But neither does it happen exclusively \textit{in} man, or decisively \textit{through} man.

Enframing is the bringing together that pertains to the settling-of which itself settles man and puts him in a position to reveal the real, by way of ordering, as standing-reserve. As the one who is exacted upon this way, man stands within the essential domain of Enframing. He is incapable of assuming a relationship to it only consequently. Thus the question of how we are to enter into a relationship to the essence of technology, asked this way, always occurs too late. But never too late is the question if we actually experience ourselves as the ones whose activities are everywhere, public and private, are exacted upon by Enframing. Moreso, never too late is the question of whether and how we actually admit ourselves into how Enframing itself exists.

The essence of modern technology starts man upon the path of the 'revealing' by which the real everywhere, more or less distinctly, becomes standing-reserve. ``To start upon the path" means ``to send" in our ordinary language. We shall call that sending-that-brings-together [\textit{versammele Schicken}] that first starts man upon a path of revelation, '\textit{destination-ing}' [\textit{Geschick}]. It is from this 'destination-ing' that the essence of all history [\textit{Geschichte}] is resolved. History is neither the mere object of written chronicle nor the mere fulfillment of human activity. Activity first becomes history as something destined (destination-ed). And it is only the 'destination-ing' into object representation that makes history accessible as an object for historiography, i.e., for a science, and on these grounds facilitates the current comparison of the historical with what is chronicled.

Enframing, as a exacting-upon into ordering, develops into a manner of revealing. Enframing is a predestining of destination-ing, like every manner of revealing. Bringing-forth, \texit{po\-{e}sis}, is also a destination-ing in this sense. The un-concealment of what is always proceeds along the 'way' of revealing. The destination-ing of revealing always governs man completely. But that destination-ing is never an impelling fate. For man is truly free only insofar as he belongs to the domain of destination-ing and so becomes one who listens and hears [\textit{H\"{o}render}], and not one who is simply obliged to obey [\textit{H\"{o}riger}].

The essence of freedom is \textit{primarily} not connected with the will or even with the provenance of human determination.

Freedom governs the open in the sense of the cleared and illuminated, i.e., of the revealed. Freedom has the closest and most intimate kinship with the 'happening' of revealing, i.e. of truth. All revealing resides within a harboring and a concealing. But that that frees -- the mystery -- is concealed and forever concealing itself. All revealing arises out of the open, goes into the open, and brings into the open. The freedom of the open resides neither in unfettered caprice nor in the constraint of mere laws. Freedom is that which conceals in a way that opens to light, in whose clearing there shimmers that veil that covers what exists of all truth and lets the veil appear as what veils. Freedom is the domain of the destination-ing that at any given time propels a revealing upon its way.

The essence of modern technology lies in Enframing. Enframing pertains to the destination-ing of revealing. These sentences express something different than the talk we frequently hear, wherein technology is the fate of our age, where ``fate" means the inevitablity of an unalterable course.

When we consider the essence of technology, we see Enframing as a destination-ing of revelation. As such, we are already lingering within the open space of destination-ing, a destination-ing that does not restrict us to a stagnant compulsion to push on blindly with technology or, equally, to rebel helplessly against it and curse it as the work of the devil. Quite the contrary, once we open ourselves expressly to the \textit{essence} of technology, we find ourselves unexpectedly ushered into a freeing claim.

The essence of technology lies within Enframing. Its governance belongs with destination-ing. Since destination-ing at any moment sets man on a path of revealing, man, thus under way, is continually approaching the brink of 1) the possibility of pursuing and pushing forward only what is revealed by ordering, and of 2) deriving all his standards from this foundation. Through this the other possibility, that man might be accepted more and sooner and ever more primally to the essence of what is revealed and to its revelation, is blocked so that he may experience his belonging to revealing as his own necessary essence.

Situated between these possibilities, man himself is endangered by way of destination-ing. The destination-ing of revelation is this way, in every one of its modes, and therefore is necessarily, \texit{danger}.

In every way the destination-ing of revelation governs, the un-concealment, in which everything that is shows itself at any moment, harbors the danger that man may flinch at the sight of the unconcealed and may come to misinterpret it. Thus where everything that manifests exhibits itself in the light of a cause-effect coherence, even God can, for representational thinking, lose all that is exalted and holy, the mysteriousness of his distance. In the light of causality, God can sink to the level of a cause, of \textit{causa efficiens}. He then becomes, even in theology, the god of the philosophers, namely, of those who define the unconcealed and the concealed in terms of the causality of making, without ever considering the essential origin of this causality.

Similarly, unconcealment--in which nature likewise presents itself as a calculable mosaic of the effects of forces--can usher in correct insights; but precisely through these successes the danger remains that in the midst of all that is correct the true will withdraw.

The destination-ing of revelation is not just \textit{any} danger, but a danger itself.

When destination-ing prevails as Enframing, it is the absolute danger. This danger reveals itself to us in two ways. As soon as what is unconcealed no longer even involves man as 'object', but does so instead exclusively as standing-reserve, and man, in the midst of objectlessness, becomes nothing but the orderer of standing-reserves, then he finds himself at the brink of a dizzying fall; that is, he has arrived at the point where he himself will be taken as standing-reserve. Meanwhile man, so threatened, glorifies himself as lord of the earth. Thus the prevailing impression becomes that everything man encounters exists only so far as it is his construct. This illusion breeds one final delusion: It appears that man only ever encounters himself. Heisenberg correctly observed that the real presents itself to modern man this way.* \texit{In truth, however, man today no longer encounters himself, i.e., his essence, anywhere}. Man stands so decisively near the exacting-forth process of Enframing that he does not understand Enframing as a claim; he fails to see himself as the one spoken to, and thus nevertheless fails to hear in what respect he ek-sists, of his essence, as an entreaty or address, and so \textit{never} encounters himself alone.

*``Das Naturbild," pp. 60 ff.

But Enframing does not just threaten man in his relationship to himself and everything that is. As a destination-ing, it banishes man to a mode of revelation that orders. Where this ordering governs, it excludes every other possible revelation. Above all, Enframing conceals revelation which, in the sense of \textit{p\-{o}esis}, permits what manifests to appear. Compared with the other mode of revelation, the 'settling-upon that exacts out' thrusts man into a liaison to what is, a relation that is at once contradictory and rigorously ordered. Where Enframing reigns, overseeing and acquiring the standing-reserve characterize all revelation. They don't even permit their own fundamental characteristic to appear, namely, the revelation itself.

Thus the exacting Enframing not only hides a prior mode of revealing, bringing-forth, it also conceals revelation itself and with it That wherein unconcealment, i.e., truth, occurs.

Enframing blocks the emanation and governance of truth. The destination-ing that accelerates into ordering is therefore extreme danger. What is dangerous is not technology. There is no demonry in technology; instead there is the mystery of its essence. The essence of technology, as a destination-ing of revelation, is the danger. The altered meaning of the word ``Enframing" will perhaps become more familiar to us now if we consider Enframing in the sense of destination-ing and danger.

The threat to man does not come initially from the potentially lethal machines and apparatus of technology. The actual threat has already affected man's essence. The predominance of Enframing threatens that man may be denied entrance into a more primal revelation and hence not experience the call of a more primal truth.

Thus, where Enframing reigns, there is \textit{danger} in the supreme sense.

> \textit{But where danger is, grows}
> \textit{The saving virtue also}

Let us think carefully about these words of H\"{o}lderlin. What does it mean ``to save"? Usually we think that it just means to grasp a thing threatened by ruin, so as to maintain it in its former state. But the verb ``to save" says more. ``To save" is to bring something home into its essence, in order to bring about the genuine appearnce of its essence for the first time. If the essence of technology, Enframing, is extreme danger, and if there is no truth in H\"{o}lderlin's words, then the reign of Enframing cannot expend itself solely in shading the en-'light'-enment of every potential revelation, all appearances of truth. Rather, technology's essence must harbor in itself the rise of the saving virtue. But in that case, might not an adequate look into what Enframing is, as a destination-ing of revelation, bring into focus the saving virtue within its arising? 

How does this saving virtue grow where the danger is?  A thing grows where it takes root, and thus it thrives. Both happen veiled and quietly and at its own leisure.  But, according to the poet's words, we have no authority to expect that we can grasp, there, immediately and without preparation, the saving virtue. Hence we must consider now, in advance, how this saving virtue deeply takes root and from there thrives even wherein the exteme danger also lies, in the governance of Enframing. In order to consider this, it is necessary, as a last step on our way, to look with yet growing vision into the danger. So, we must once again question concerning technology. For we have said that in technology's essence roots and thrives the saving virtue.

But how shall we discern the saving virtue in the essence of technology if we do not consider in what sense of ``essence" it is that Enframing is in truth the essence of technology?

So far we have understood ``essence" in its current meaning. In the academic language of philosophy, ``essence" menas \texit{what} something is; in Latin, \textit{quid}. \texit{Quidditas}, whatness, provides the answer to the question concerning essence. For example, what belongs to all kinds of trees -- oaks, beeches, birches, firs -- is the same ``treeness". Under this comprehensive subdivision -- the ``universal" -- fall all real and possible trees. Is then the essence of technology, Enframing, the common family for everything technological? If that were the case then the steam turbine, the radio transmitter, and the cyclotron would each be an Enframing. But here the word ``Enframing" does not mean a tool or kind of equipment. Even less does it imply the general notion of such resources. The machines and equipment are no more types and kinds of Enframing than is the man at the switchboard or the engineer in the drafting room. Each of these, in its own way, does belong as inventory, available resource, or executer, within Enframing; but Enframing is not the essence of technology in the sense of a genus. Enframing is a manner of revealing the possession of the character of destination-ing, namely, the course that exacts. The revelation that produces (\textit{p\-{o}esis}) is also a mode that has the character of destination-ing. But these modes are not types that, once aligned, fall under the theory of revelation. Revealing is that destination-ing which, ever suddenly and inexplicably to all meditation, bestows itself upon that revealing which conceives and also exacts, and which earmarks itself for man. The exacting revelation begins as a destination-ing in producing. But at the same time Enframing, in that characteristic manner of destination-ing, blocks \textit{p\-{o}esis}.

Thus Enframing, as a destination-ing of revelation, is indeed the essence of technology, but not in the sense of genus and \textit{essentia}. As we take note of this, something astounding strikes us: It is technology itself that compels us to reconsider what is usually understood by ``essence". But in what way?

When we speak of the ``essence of a house" or the ``essence of a state," we do not mean a generic type; rather we mean the ways in which houses and states govern, administer themselves, develop and decay - the way in which they ``essence" [\textit{Wesen}]. Johann Peter Hebel in a poem, ``Ghost on Kanderer Street," for which Goethe had a special fondness, uses the old word \textit{die Weserei}. It means the city hall in the sense that there the life of the community gathers and there village existence is constantly in play, i.e., comes to presence. The noun is derived from the verb \texit{wesen}. \textit{Wesen}, as a verb, is the same as \textit{w\"{a}hren} [to last or endure] not only in meaning, but also in the phonetic formation of the word. Socrates and Plato already consider the essence of a thing as what essences, what comes to presence, what endures. But they consider what endures to be what remains permanently (\textit{aei on}). They find what endures permanently in what remains, what tenaciously persists throughout all. What remains is discovered, in turn, in the aspect (\textit{eidos, idea}), for example, the Idea ``house."

The Idea ``house" demonstrates what it is that is intended as a house. Individual, real, and possible houses, on the other hand, are dynamic and transitory derivatives of the Idea and thus belong to what does not endure.

But it can not be established that endurance is based only on what Plato considered \textit{idea} and Aristotle considered \textit{to ti \-{e}n einai} (that which any thing has always been), or what metaphysics' varied interpretations considers \textit{essentia}.

All essencing endures. But is all endurance permanent? Does the essence of technology endure in the sense of a permanently enduring Idea that hovers over all things technological, making it such that by 'technology' we mean some mythological abstraction? The way technology essences reveals itself only in that permanent enduring in which Enframing comes to pass as a destination-ing of revealation. Goethe once used the mysterious word \textit{fortgew\"{a}hren} [to grant permanently] in place of \textit{fortw\"{a}hren} [to endure permanently].* We hear \textit{w\"{a}hren} [to endure] and \textit{gew\"{a}hren} [to grant] here in one unarticulated accord. And if we now more carefully consider what it is that actually, and perhaps alone, endures, we may venture to say: \textit{Only what is granted endures. What originally endures out of the earliest beginning is what, in turn, grants}. 

*``Die Wahlverwandtschaften" [Congeniality], pt II, chap. 10, in the novelette \textit{Die wunderlichen Nachbaskinder} [The strange neighbor's children].

As the essence-ing of technology, Enframing is what endures. Does Enframing's governance extend into the sense of granting? No doubt the question appears a horrendous blunder. According to everything that has been said, Enframing is, rather, a destination-ing that brings into the revelation that exacts. Challenging is anything but a granting. So it seems, so long as we do not notice that the challenging-forth into the ordering of the real as standing-reserve still remains a destination-ing that puts man upon a path of revealing. As destination-ing, the manifestation of technology grants man entry into That which, of himself, he can neither invent nor make. There is no such thing as a man who, wholly of himself, is only man.

But if this destination-ing, Enframing, is the extreme danger, not only for man's manifestation, but for all revelation as such, can this destination-ing still be called a granting? Most emphatically yes, if in this destination-ing the the saving virtue grows. Every relevatory destination-ing occurs as a result of a granting and as a granting. For it is granting that first conveys to man his share in revelation which the occurrence of revelation needs. As the one so required and used, man is inclined to belong to the occurrence of truth. This granting that pushes in one way or another into revealing is, as such, the saving virtue. For the saving virtue permites man to see and enter into the highest dignity of his own essence. This dignity is in sheparding in the un-concealment -- and with it, primally, the concealment -- of all coming to presence on this earth. It is precisely in Enframing, which threatens to sweep man away into ordering as the singular mode of revealing, and so puts man into the danger of surrendering his free essence -- it is precisely in this extreme danger that the innermost indestructible inclusion of man within granting comes to light, provided that we, for our part, take heed of the arrival of technology.

Thus the mainfestation of technology harbors in itself what we least suspect, the possible birth of the saving virtue. 

Everything, then, depends upon this: that we ponder this birth and that, remembering, we watch over it. How can we do this? Above all by catching sight of what manifests in technology, instead of merely staring at the technological. As long as we represent technology as an instrument, we remain caught in the will to master it. We press on past the essence of it.

When, however, we ask how the instrumental manifests as a form of causality, then we perceive manifestation as the destination-ing of a revelation.

When we consider, finally, that the manifestation of the essence of technology occurs within the granting which requires and uses man so that he may participate in the revelation, then the following becomes clear:

The essence of technology is, in a lofty sense, ambiguous. Such ambiguity augurs the mystery of all revelation, i.e., of truth.

On the one hand, Enframing exacts the frenziedness of ordering that obscures every view of the occurence-ing of revelation and so gravely endangers its relation to the essence of truth.

On the other, Enframing, for its part, happens in the granting that allows man to endure -- as yet unexperienced, but perhaps experienced in the future -- so that he may be the one who is required and used for the safekeeping of truth's manifestation. Thus does the origination of the saving virtue appear.

The irresistibility of ordering and the restraint of the saving virtue fly past each other like the paths of two stars in the heavens. But precisely this, their passing by, is a hidden aspect of their nearness.

When we investigate the ambiguous essence of technology, we behold the constellation, the stellar course of the mystery.

The question concerns the constellation in which revealing and concealing, where truth's manifestation, comes to pass.

But what use is it to us to look into the constellation of truth? We look into the danger and see the roots of the saving virtue.

Through this, however, we are not yet saved. But we are therefore summoned to hope in the birthing light of the saving virtue. How can this happen? Here and now and in little actions, that we may nurture the saving virtue in its development. This entails holding always before our eyes the extreme danger.

The manifestation of technology jeopardizes revealing, threatens it with the possibility that all revealing will be consumed in ordering and that everything will present itself only as standing-reserve. Human activity cannot directly counter this danger. Human achievement alone cannot banish it. But human reflection can ponder the fact that all saving virtue must be of a higher essence than what is endangered, while simultaneously kindred to it.

But might perhaps there be a more fundamentally granted revelation that could bring the saving virtue into its first shining forth in the midst of the danger, a revelation which, in our technological age, conceals rather than shows itself?

There was a time when technology was not alone in bearing the name \textit{techn\-{e}}. Once the revealing which brings forth truth into the beautiful was also called \textit{techn\-{e}}.

Once there was a time when the bringing-forth of the true into the beautiful was called \textit{tech\-{e}}. And the \textit{po\-{e}sis} of fine arts also was called \textit{tech\-{e}}.

In Greece, at the beginning of the destination-ing of the West, the arts soared to the supreme height of the revelation granted them. They brought the presence of the gods, brought the dialogue of divine and human destinings, to radiance. And art was simply called \textit{techn\-{e}}. It was a single, abundant revelation. It was pious, \textit{promos}, i.e., yielding to the authority and the safekeeping of truth.

The arts were not derived from the artistic. Art works were not enjoyed aesthetically. Art was not a sector of cultural activity.

What, then, was art--perhaps only for that brief but magnificent time? Why did art bear the modest name \textit{techn\-{e}}? Because it was a revelation that brought forth and hither, and therefore belonged within \texit{po\-{e}sis}. It was finally that revealing which holds complete sway in all the fine arts, in poetry, and in everything poetical that attained \textit{po\-{e}sis} as its proper name.

The same poet from whom we heard the words

>\textit{But where danger is, grows}
>\textit{The saving also,}

says to us:

>\textit{...poetically dwells man upon this earth.}

Poetic language delivers the true into the splendor of what Plato, in the \textit{Phaedrus}, calls \textit{ekphanestaton}, that that blazes forth most pure. The poetic thoroughly pervades every art, every revealing of the manifestation of beauty.

Could it be that fine arts are called to poetic revelation? Could it be that revelation fundamentally lays claim to the arts, so that they for their part may expressly foster the growth of the saving virtue, may awaken and form anew our attention into that which grants and our trust in it?

Whether art may be granted this highest responsibility of its essence in the midst of the extreme danger, no one can tell. Yet we can be astounded. Before what? Before this other possibility: that the frenziedness of technology may entrench itself everywhere to such an extent that someday, throughout everything technological, the essence of technology may manifest in the coming-to-pass of truth.

Because the essence of technology is nothing technological, essential reflection upon technology and decisive confrontation with it must happen in a realm that is, on the one hand, akin to the essence of technology and, on the other, fundamentally different from it.

Such a realm is art. But only if reflection on art, for its part, does not close its eyes to the constellation of truth after which we are \textit{questioning}.

Thus questioning, we bear witness to the crisis that in our sheer preoccupation with technology we do not experience the manifestation of technology, that in our sheer aesthetic-mindedness we no longer protect and preserve the manifestation of art. Yet the more questioningly we ponder the essence of technology, the more mysterious the essence of art becomes.

The closer we come to the danger, the more brightly do the paths of the saving virtue begin to shine and the more questioning we become. For questioning is the holiness of thought.

\end{document}
